\documentclass{article}

\usepackage[spanish]{babel}
\usepackage{amsmath}
\usepackage[utf8]{inputenc}

\title{Método Simplex}
\author{Penélope W. Gutiérrez Silva}

\begin{document}
\maketitle

\section{Introducción:}

El método simplex es un algoritmo para resolver problemas de programación lineal. Fue desarrollado por George Dantzing en 1947.

\section{Ejemplo:}

Ilustraremos la aplicación del método simplex con un ejemplo.

Resuelve el siguiente problema:
 
      \begin{equation*}
 \begin{aligned}
\text{Maximizar} \quad & 2x_{1}+x_{2}\\
\text{sujeto a} \quad &
  \begin{aligned}
   x_{1}-x_{2} &\leq 2\\
   -2x_{1}+x_{2} &\leq 2\\
   3x_{1}+4x_{2} &\leq 12\\
   x_{1}+x_{2} &\geq 1\\
    x_{1},x_{2} &\geq 0
  \end{aligned}
\end{aligned}
\end{equation*}
 
Como una de las desigualdades aparecen las variables  del lado izquierdo de un símbolo $\leq$, multiplicamos ambos miembros de esa desigualdad por $-1$.

     \begin{equation*}
 \begin{aligned}
\text{Maximizar} \quad & 2x_{1}+x_{2}\\
\text{sujeto a} \quad &
  \begin{aligned}
   x_{1}-x_{2} &\leq 2\\
   -2x_{1}+x_{2} &\leq 2\\
   3x_{1}+4x_{2} &\leq 12\\
   -x_{1}-x_{2} &\leq -1\\
    x_{1},x_{2} &\geq 0
  \end{aligned}
\end{aligned}
\end{equation*}

Para obtener la forma simplex, añadimos una variable de holgura por cada desigualdad de forma que se conviertan en igualdades.

\begin{equation*}
 \begin{aligned}
\text{Maximizar} \quad & 2x_{1}+x_{2}\\
\text{sujeto a} \quad &
  \begin{aligned}
   x_{1}-x_{2}+X_{3} &= 2\\
   -2x_{1}+x_{2}+X_{4}&= 2\\
   3x_{1}+4x_{2}+X_{5}&= 12\\
   -x_{1}-x_{2}+X_{6}&= -1\\
   x_{1},x_{2},X_{3},X_{4},X_{5},X_{6} &\geq 0
  \end{aligned}
\end{aligned}
\end{equation*}


A continuación obtenemos un \emph{tablero simplex} despejando las variables de holgura.



\end{document}
