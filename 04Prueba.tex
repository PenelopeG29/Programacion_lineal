
\documentclass{article}

\usepackage[utf8]{inputenc}
\usepackage{amsmath}
\usepackage[spanish]{{babel}}

\title{Apuntes de programación lineal}

\author{Penélope Wendoly Gutiérrez Silva}


\begin{document}

\maketitle
\tableofcontents

\section{Introducción}
\label{sec:introduccion}




La forma estándar de un problema de progamación lineal es:

Dados una matriz $A$ y vectores $b,c$

maximizar $c^{T}x$

sujeto a $Ax\leq b$ y $x \geq 0$

\smallskip

la forma ...

\medskip

la forma...

\bigskip

la forma...

\subsection{Tablas}

\begin{tabular}{|c|c|c|}
\hline
  & A & B  \\
  \hline
  Máquina 1 & 1&2\\
  \hline
  Máquina 2& 1&1\\
  \hline
  \end{tabular}

\subsection{Matrices}
\begin{equation}
  \label{eq:1}
  A=
  \begin{pmatrix}
    0 & 1 & 2\\
    3 & -1 & 5\\
    9 & 3 & 6
  \end{pmatrix}
  +
  \begin{pmatrix}
    0& 1 & 1\\
    2 & 4 & 8\\
    3 & 6 & 5 
   \end{pmatrix}
\end{equation}

\end{document}